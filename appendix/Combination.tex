\chapter{组合计数}

全部组合分析公式的推导基于下列两条原理:
\begin{description}
  \item[乘法原理] 若进行$A_1$过程有$n_l$种方法,进行$A_2$过程有 $n_2$种方法,则进行$A_1$过程后再接着进行$A_2$程共有$n_1 \cdot  n_2$种方法
  \item[加法原理] 若进行$A_1$过程有$n_l$种方法,进行$A_2$过程有 $n_2$种方法,假定$A_1$过程与$A_2$过程是并行的,则进行过程$A_1$或过程$A_2$的方法共有$n_1 +n_2$种
\end{description}

排列与组合的定义及其计算公式如下:
\begin{description}
  \item[排列] 从$n$个不同元素中任取$r (r \le n)$个元素排成一列(考虑元素先后出现次序),称此为一个排列,其总数记为$P_n^r$。按乘法原理有:
    \[ P_n^r = n \times (n - 1) \times \dotsb \times (n - r + 1) = \frac{n!}{(n - r!)} \]
    若$r = n$,则称为全排列,记为 $_n$。显然,全排列 $P_n = n!$。
  \item[重复排列] 从$n$个不同元素中每次取出一个,\underline{放回后}再取下一个,如此连续取$r$次所得的排列称为重复排列。此种重复排列数共有$n^r$个。
  \item[组合] 从$n$个不同元素中任取$r (r \le n)$个元素并成一组(不考虑元素间的先后次序),称此为一个组合,其总数记为$\binom{n}{r}$或$C_n^r$。按乘法原理有:
    \[ \binom{n}{r} = \frac{P_n^r}{r!} = \frac{n (n - 1) \dotsb (n - r + 1)}{r!} = \frac{n!}{r! (n - r)!} \]
    在此规定 $0! = 1$ 与 $\binom{n}{0} = 1$.
  \item[重复组合] 从$n$个不同元素中每次取出一个,\underline{放回后}再取下一个,如此连续取$r$次所得的组合称为重复组合.其总数为$\binom{n + r - 1}{r}$。注意这里的 $r$ 也允许大于 $n$.
\end{description}

重复组合数的得出可如下考虑:将$n$个元素看作$n$个盒子,使用$n+1$个插板区分(相邻两插板组成一个盒子),取$r$次视为往盒子中放入$r$个球。每种取法可视为将$n-1$个插板(两头的不能动)与$r$个球进行放置。共有$r+n-1$个位置,但只要其中的插板(或球)放置好后,球(或插板)的位置自然固定。所以其总数为$\binom{n + r - 1}{n-1}$(或$\binom{n + r - 1}{r}$,两者相同)。

\begin{table}[h]
  \centering
  \begin{tabular}{|c|c|c|}
    \hline
                     & \bfseries 有序            & \bfseries 无序                 \\ \hline
    \bfseries 放回   & 重复排列$n^r$             & 重复组合$\binom{n + r - 1}{r}$ \\ \hline
    \bfseries 不放回 & 排列$\frac{n!}{(n - r!)}$ & 组合 $\binom{n}{r}$            \\ \hline
  \end{tabular}
\end{table}

\begin{theorem}[牛顿二项式定理]\label{No}
  若对于任意实数$\alpha$定义
  \[ \binom{\alpha}{r}=\frac{A^{\alpha}_{r}}{r!}=\frac{\alpha(\alpha-1)\cdots (\alpha-r+1)}{r!} \]
  则有牛顿二项式:
  \[ (1+x)^{\alpha}=\sum_{r=0}^{\infty}\binom{\alpha}{r}x^r \]
\end{theorem}

\begin{proof}
  %TODO
\end{proof}