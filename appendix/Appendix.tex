\chapter{基本数学工具}

\section{集合论与测度论}

\section{排列与组合}

全部组合分析公式的推导基于下列两条原理:
\begin{description}
  \item[乘法原理] 若进行$A_1$过程有$n_l$种方法,进行$A_2$过程有 $n_2$种方法,则进行$A_1$过程后再接着进行$A_2$程共有$n_1 \cdot  n_2$种方法
  \item[加法原理] 若进行$A_1$过程有$n_l$种方法,进行$A_2$过程有 $n_2$种方法,假定$A_1$过程与$A_2$过程是并行的,则进行过程$A_1$或过程$A_2$的方法共有$n_1 +n_2$种
\end{description}

排列与组合的定义及其计算公式如下.
\begin{description}
  \item [排列]  从 $n$ 个不同元素中任取 $r (r \le n)$ 个元素排成一列 (考虑元素先后出现次序), 称此为一个排列,此种排列的总数记为 $P_n^r$, 按乘法原理,取出的第一个元素有 $n$ 种取法,取出的第二个元素有 $n - 1$ 种取法\dots 取出的第 $r$ 个元素有 $n - r + 1$ 种取法, 所以有
        \[ P_n^r = n \times (n - 1) \times \dotsb \times (n - r + 1) = \frac{n!}{(n - r!)} \]
        若 $r = n$, 则称为全排列, 记为 $_n$. 显然,全排列 $P_n = n!$.
  \item [重复排列]  从 $n$ 个不同元素中每次取出一个,放回后再取下一个,  如此连续取 $r$ 次所得的排列称为重复排列, 此种重复排列数共有 $n^r$ 个.注意这里的 $r$ 允许大于 $n$.
  \item [组合]   从 $n$ 个不同元素中任取 $r (r \le n)$ 个元素并成一组 (不考虑元素间的先后次序), 称此为一个组合, 此种组合的总数记为 $\binom{n}{r}$ 或 $C_n^r$. 按乘法原理此种组合的总数为
        \[ \binom{n}{r} = \frac{P_n^r}{r!} = \frac{n (n - 1) \dotsb (n - r + 1)}{r!} = \frac{n!}{r! (n - r)!} \]
        在此规定 $0! = 1$ 与 $\binom{n}{0} = 1$.
  \item [重复组合] 从 $n$ 个不同元素中每次取出一个, 放回后再取下一个, 如此连续取 $r$ 次所得的组合称为重复组合, 此种重复组合总数为 $\binom{n + r - 1}{r}$. 注意这里的 $r$ 也允许大于 $n$.
\end{description}

上述四种排列组合及其总数计算公式,在确定概率的古典方法中经常使用,但在使用中要注意识别有序与无序、重复与不重复。

\begin{theorem}[牛顿二项式定理]\label{No}
  若对于任意实数$\alpha$定义
  \[ \binom{\alpha}{r}=\frac{A^{\alpha}_{r}}{r!}=\frac{\alpha(\alpha-1)\cdots (\alpha-r+1)}{r!} \]
  则有牛顿二项式:
  \[ (1+x)^{\alpha}=\sum_{r=0}^{\infty}\binom{\alpha}{r}x^r \]
\end{theorem}

\begin{proof}
  %TODO
\end{proof}