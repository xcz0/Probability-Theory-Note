\chapter{测度论基础}

简而言之,测度论可以理解为在抽象空间建立类似于实变函数中测度、积分和导数那样的分析系统。

\section{可测空间和可测映射}

\subsection{集合及其运算}

考虑一个任意非空集合$X$,称之为\textbf{空间}。$X$的子集以$A.B,C,\cdots $等记之,称之为这个空间的\textbf{集合}。空集记为$\emptyset$。$X$的成员称为\textbf{元素}。元素$x$属于集合$A$,记作$x \in A$;反之,元素$x$不属于集合$A$,则用记号$x \notin A$来表示。

\begin{definition}
    空间$X$上定义的实函数
    \[ I_{A}(x)=\begin{cases}
            1, \quad x \in A \\
            0, \quad x \notin A
        \end{cases} \]
    称为$A$的\textbf{指示函数}。
\end{definition}

\begin{definition}[集合的运算]
    给定集合$A$和$B$,集合
    \begin{description}
        \item[余] $A^{\complement} := \{ x \in X | x \notin A \}$
        \item[并] $A \cup B := \{ x | (x \in A)\vee(x \in B) \}$
        \item[交] $A \cap B := \{ x | (x \in A)\wedge(x \in B) \}$
        \item[差] $A \setminus B := \{ x | (x \in A)\wedge(x \notin B) \}$
        \item[对称差] $A \bigtriangleup B := (A \setminus B)\cup (B \setminus A)$
    \end{description}
\end{definition}

\begin{definition}[集合的极限]
    设$\{ A_n \}$是一个集合序列,
    \begin{enumerate}
        \item 若
              \[ \forall n \in N_+ ,\quad A_{n+1} \subseteq A_n \]
              则称$\{ A_n \}$为\textbf{非降}的,记为$A_n\uparrow$, 并称
              \[ \lim_{n \to \infty}A_n := \bigcup_{n=1}^{+\infty}A_n \]
              为它的\textbf{极限}.
        \item 若
              \[ \forall n \in N_+ ,\quad A_{n} \subseteq A_{n+1} \]
              则称$\{ A_n \}$为\textbf{非增}的,记为$A_n\downarrow$, 并称
              \[ \lim_{n \to \infty}A_n := \bigcap_{n=1}^{+\infty}A_n \]
              为它的\textbf{极限}.
    \end{enumerate}
    非降或非增的集合序列统称为\textbf{单调序列}.因此,\underline{单调集合序列总有极限}.
\end{definition}

\begin{definition}[上极限与下极限]
    对于任意给定的一个集合序列$\{ A_n \}$, 集合序列$\{B_n= \bigcap_{k=n}^{+\infty}A_k \}$与$\{B'_n=  \bigcup_{k=n}^{+\infty}A_k \}$分别是非降和非增的,因而分别有极限:
    \begin{align*}
        \varliminf_{n \to \infty}A_n =\lim_{n \to \infty} B_n = \bigcup_{n=1}^{\infty} B_n= \bigcup_{n=1}^{\infty} \bigcap_{k=n}^{+\infty}A_k \\
        \varlimsup_{n \to \infty}A_n =\lim_{n \to \infty} B'_n = \bigcap_{n=1}^{\infty} B'_n= \bigcap_{n=1}^{\infty} \bigcup_{k=n}^{+\infty}A_k
    \end{align*}
    分别称其为$\{ A_n \}$的\textbf{上极限}与\textbf{下极限}
\end{definition}

显然,记号$x \in \varlimsup_{n \to \infty}A_n $意味元素$x$属于序列$\{ A_n \}$中的无穷多个集合,而记号$x \in \varliminf_{n \to \infty}A_n $,则表明除去$\{ A_n \}$中的有限个集合外,元素$x$属于该序列的其余集合.于是我们有
\[ \varliminf_{n \to \infty}A_n \subseteq  \varlimsup_{n \to \infty}A_n \]

\begin{definition}
    如果$\varliminf_{n \to \infty}A_n = \varlimsup_{n \to \infty}A_n$, 则认为$\{ A_n \}$的极限存在, 并且将
    \[ \lim_{n \to \infty}A_n := \varliminf_{n \to \infty}A_n = \varlimsup_{n \to \infty}A_n \]
    称为它的\textbf{极限}.
\end{definition}

\subsection{集合系}

