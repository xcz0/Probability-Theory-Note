\chapter{概率基础}

\begin{introduction}
    \item 事件
    \item 古典概型与几何概率
    \item 条件概率与独立
    \item 乘法法则
    \item 全概率公式
    \item Bayes法则
\end{introduction}

\section{概率空间}

\begin{definition}[样本空间]
    随机试验可能出现的结果称为\textbf{样本点}(sample point),用$\omega$表示。样本的全体构成\textbf{样本空间}(sample space),用$\Omega$表示。
\end{definition}

\begin{definition}[事件的古典定义]
    样本点$\omega$的集合称为\textbf{事件}(event)。
\end{definition}

我们关心的随机现象被抽象为集合, 逻辑运算(且, 或, 非, etc.)对应成集合论运算(交, 并, 补, etc.)。

\begin{property}
    集合的运算性质:
    \begin{description}
        \item [交换律] $A \cup B = B \cup A, \quad AB = BA$
        \item [结合律] $(A \cup B) \cup C = A \cup (B \cup C), (AB)C = A(BC)$
        \item [分配律]
              \begin{gather}
                  (A \cup B) \cup C = A \cup (B \cup C),\\
                  (A \cap B) \cup C = (A \cup C) \cap (B \cup C).
              \end{gather}
        \item [对偶律(De Morgan's laws)]
              \begin{gather}
                  \text{事件并的对立等于对立的交:} \quad \overline{A \cup B} = \overline{A} \cap \overline{B},\\
                  \text{事件交的对立等于对立的并:} \quad \overline{A \cap B} = \overline{A} \cup \overline{B}.
              \end{gather}
    \end{description}
\end{property}

为方便概率的定义,并不把$\Omega$的一切子集作为事件,应避免不可测集的出现。

\begin{definition}
    事件构成的全体称为\textbf{事件域}$\mathscr{F}$,是$\Omega$的子集族(collection of subsets),应满足\underline{\,$\sigma$代数}的要求:
    \begin{itemize}
        \item $\Omega \in \mathscr{F}$, 样本空间的整体可作为一个事件;
        \item $A\in\mathscr{F} \implies A^{\complement}\in\mathscr{F}$, 即$\mathscr{F}$对补集运算(逻辑上的非)封闭;
        \item $A_{1},\dots,A_{n},\ldots \in \mathscr{F} \implies \bigcap_{n=1}^{\infty}A_{n} \in \mathscr{F}$, 即$\mathscr{F}$对可数交运算(逻辑上的可数多个且)封闭.
    \end{itemize}
\end{definition}

\begin{note}
    可数性是为了在数学上能够恰当地处理\underline{无穷}的概念, 术语中的$\sigma$指的就是\underline{可数并}。由对偶原理可得$\sigma$域同时对可数并运算封闭. 即$\sigma$域对逆, 并, 交, 差的可数次运算封闭.
\end{note}

事件域根据问题的不同要求适当选取, 在概率定义没有困难时, 应尽量选大, 通常以$\Omega$的一切子集作为事件域. 当$\Omega$给定后,若某些子集必须作为事件处理, 能否找到包含他们的$\sigma$域?

\begin{proposition}
    若给定$\Omega$的一个非空集族$\mathscr{G}$, 必存在$\Omega$上唯一的$\sigma$域$\mathfrak{m}(\mathscr{G})$, 满足下列性质:
    \begin{itemize}
        \item 包含$\mathscr{G}$
        \item 若其他$\sigma$域包含$\mathscr{G}$, 则必包含$\mathfrak{m}(\mathscr{G})$
    \end{itemize}
    这个$\mathfrak{m}(\mathscr{G})$称为包含$\mathscr{G}$的最小$\sigma$域, 或由$\mathscr{G}$产生的$\sigma$域.
\end{proposition}

\begin{proof}
    由于$\Sigma$的一切子集构成的集类包含$\mathscr{G}$, 所以$\mathfrak{m}$存在. 再取$\Sigma$上满足此条件的$\sigma$域之交作为$\mathfrak{m}(\mathscr{G})$即可.
\end{proof}

\begin{definition}
    设$\mathbb{R}^1$为全集, 形为$[a,b)$构成的集类产生的$\sigma$域称为\textbf{一维Borel$\sigma$域}, 记为$\mathscr{B}_1$, 其中的元素称为\textbf{一维Borel集}
\end{definition}

若$x,y$为任意实数,由于:
\begin{align*}
    \{x\}  & =  \bigcap_{n=1}^{\infty}\left[x, x+\frac{1}{n}\right) \\
    (x, y) & =  [x, y)-\{x\}                                        \\
    [x, y] & =  [x, y)+\{y\}                                        \\
    (x, y] & =  [x, y)+\{y\}-\{x\}
\end{align*}
因此$\mathscr{B}_1$包含一切开区间, 闭区间, 单个实数, 可列个实数, 以及他们经可列次逆, 并, 交, 差运算的集合.

\begin{definition}
    定义在事件域上的集合函数$P : \mathscr{F} \to \mathbb{R}$称为\textbf{概率}的条件是:
    \begin{description}
        \item[非负性] $P(A)\ge 0 , \forall A \in \mathscr{F}$
        \item[规范性] $P(\Omega) = 1$; (如果没有这条就是一般的{\color{lightgray}有限}\emph{测度})
        \item[可列可加性] 若$A_{1},\dots,A_{n},\ldots \in \mathscr{F}$两两不交, 即$A_{i}\cap A_{j} = \emptyset, \ \forall i\neq j$, 则$P(\bigcup_{n=1}^{\infty}A_{n}) = \sum_{n=1}^{\infty}P(A_{n})$.
    \end{description}
    我们称$(\Omega,\mathscr{F},P)$为一个\textbf{概率空间}(probability space)
\end{definition}

\begin{corollary}
    $P(\bar{A})=1-P(A)$
\end{corollary}

\begin{corollary}
    $ P(A+B)=P(A)+P(B)-P(AB)$
\end{corollary}

\begin{corollary}
    \begin{align*}
         & P\left(A_{1} \cup A_{2} \cup \cdots \cup A_{n}\right)=\sum_{i=1, \cdots, n} P\left(A_{i}\right)-\sum_{\substack{i<j \\
        i, j=1, \cdots, n}} P\left(A_{i} A_{j}\right)                                                                          \\
         & +\sum_{\substack{i<j<k                                                                                              \\
                i, j, k=1, \cdots, n}} P\left(A_{i} A_{j} A_{k}\right)-\cdots+(-1)^{n-1} P\left(A_{1} A_{2} \cdots A_{n}\right)
    \end{align*}
    特别地, 若事件出现个数相同时概率相等,则可简化为:
    \[ P\left(A_{1} \cup A_{2} \cup \cdots \cup A_{n}\right)=n P_{1} - \binom{n}{2} P_{2} + \binom{n}{3} P_{3} \cdots+(-1)^{n-1} P_{n} \]
\end{corollary}



\section{古典概型与几何概率}

\section{条件概率}

考虑正概率的事件$A \in \mathscr{F}$, 称
\[ P(\bullet|A) = P(\bullet\cap{A})/P(A), \quad \bullet \in \mathscr{F} \]
为\textbf{条件于$A$的概率}(probability conditional on $A$), 这仍然是一个概率测度.

% 如果$A,B\in\mathscr{F}$满足
% \[ P(A\cap B) = P(A)P(B), \]
% 则称$A$与$B$\textbf{独立}(independent), 记为$A\indep B$. 易见独立性是对称的, 即$A\indep B \iff B\indep A$.

% 当$P(A)>0$时, 我们有
% \[ P(B|A) = P(B) \iff B\indep A, \]
% 由此可得到``$B$独立于$A$''的直观理解.

% 若$\mathscr{G} \subset \mathscr{F}$与$\mathscr{H} \subset \mathscr{F}$满足
% \[ A \indep B, \quad \forall A\in\mathscr{G},\ B\in\mathscr{H}, \]
% 则称$\mathscr{G}$与$\mathscr{H}$独立, 记为$\mathscr{G} \indep \mathscr{H}$.

%测度论告诉我们一个重要结果: 如果$\mathscr{G}$对交集运算封闭, 那么成立$\mathscr{G}\indep\mathscr{H} \implies \sigma(\mathscr{G}) \indep \mathscr{H}$, 其中$\sigma(\mathscr{G})$是$\mathscr{G}$扩张而成的一个合适的最小的$\sigma$代数, 定义为所有包含$\mathscr{G}$的$\sigma$代数的交集.

\emph{扩张}, 或者称为\emph{延拓}, 是数学中很重要的一个概念, 大抵是将某映射的定义域适当扩大, 不改变在初始定义域上的映射取值(注意值域可能是比较抽象的集合, 配备了某些操作之后被称为空间), 同时在扩大后的定义域上仍然保持某些优良的性质. 与此相对的概念是\emph{限制}, 即关心局部上可能更加漂亮的性质, 把初始的定义域适当缩小.

