\newcommand{\ucite}[1]{\textsuperscript{\cite{#1}}}
% 参考文献引用:上标用 \ucite{ }, 文中用 \cite{ }
\newcommand{\red}[1]{{\color{red}#1}}

\newcommand{\widesim}[2][1.5]{
	\mathrel{\overset{#2}{\scalebox{#1}[1]{$\sim$}}}
}
\newcommand{\iidsim}{\widesim[2.33]{\mathrm{i.i.d.}}}	% independent and identically distributed
\newcommand{\indep}{\Vbar}		% independent
\newcommand{\X}{\mathcal{X}}		% 
\newcommand{\1}{\mathbbm{1}}				% indicator
\renewcommand{\P}{\mathbb{P}}				% probability
\newcommand{\as}{\mathrm{a.s.}}				% almost surely
\newcommand{\E}{\mathbb{E}}					% expectation
\newcommand{\Var}{\operatorname{Var}}		% variance
\newcommand{\Cov}{\operatorname{Cov}}		% covariance
\newcommand{\Corr}{\operatorname{Corr}}		% correlation
\newcommand{\MSE}{\operatorname{MSE}}		% mean square error
\newcommand{\mean}[1]{\overline{#1}}
\newcommand{\abs}[1]{\left\lvert#1\right\rvert}		% absolute value
\newcommand{\norm}[1]{\left\lVert#1\right\rVert}
\newcommand{\rank}[1]{\operatorname{rank}\left(#1\right)}
\newcommand{\tr}[1]{\operatorname{tr}\left(#1\right)}		% trace
\newcommand{\diag}[1]{\operatorname{diag}\!\left(#1\right)}			% diagonal
\newcommand{\vspan}[1]{\operatorname{span}\left(#1\right)}	% vector span
\newcommand{\proj}[2]{\operatorname{proj}_{#2}#1}			% projection
\newcommand{\inprod}[2]{\left\langle #1, #2 \right\rangle}	% inner product
\newcommand{\R}{\mathbb{R}}					% real
\renewcommand{\Rn}{\mathbb{R}^n}				% real n dimensions
\renewcommand{\d}{\mathrm{d}}				% differential
\newcommand{\pd}{\partial}					% partial differential
\newcommand{\p}[3][]{\frac{\pd^{#1}#2}{\pd{#3}^{#1}}}	% partial derivative
\newcommand{\e}{\varepsilon}
\newcommand{\st}{,\quad \mathrm{s.t.}\ }			% subject to, so that
\newcommand{\argmin}{\operatornamewithlimits{arg\,min}}
\newcommand{\argmax}{\operatornamewithlimits{arg\,max}}
\newcommand{\N}{\mathcal{N}}				% normal, not natural numbers XD
\newcommand{\ee}{\mathrm e}
\newcommand{\dd}{\mathop{}\!\mathrm d}
\newcommand{\textop}[1]{\relax\ifmmode\mathop{\text{#1}}\else\text{#1}\fi}
\newcommand{\TT}{^{\mathrm T}}
\newcommand{\U}{\mathrm{U}}

\renewcommand\arraystretch{2}

\setlist[description]{
	labelindent = 3em,
	labelwidth = 7em,
	leftmargin = 10em,
	labelsep = 0em,
	parsep = 0.75em,
	topsep = 0.75em,
}

\tikzset{
box/.style ={
rectangle, %矩形节点
rounded corners =5pt, %圆角
minimum width =50pt, %最小宽度
minimum height =20pt, %最小高度
inner sep=5pt, %文字和边框的距离
draw=blue %边框颜色
}}